\begin{center}
%{\huge (Septiembre, 2025)}
\end{center}

\setlength{\LTleft}{-1.25cm}      % Desplaza la tabla 1cm a la izquierda
\setlength{\LTright}{0pt}      % Mantiene el margen derecho normal

\begin{longtable}{|l|l|}
	\cline{1-2}
	\multicolumn{2}{|c|}{}	\\
	\multicolumn{2}{|c|}{\textbf{UNIVERSIDAD DE LEÓN}}	\\
	\multicolumn{2}{|c|}{\textbf{Departamento de Ingenierías Mecánica, Informática y Aeroespacial}}	\\ 
	\multicolumn{2}{|c|}{\textbf{MÁSTER UNIVERSITARIO EN ROBÓTICA E INTELIGENCIA ARTIFICIAL}}	\\
	\multicolumn{2}{|c|}{\textbf{Trabajo de Fin de Máster}}	\\ 
	\multicolumn{2}{|c|}{}	\\ \hline
	\multicolumn{2}{|l|}{\textbf{ALUMNO}: Aitor García Blanco}	\\ \hline
	\multicolumn{2}{|l|}{\textbf{TUTOR}: Laura Fernández Robles}	\\ \hline
	\multicolumn{2}{|p{17cm}|}{\textbf{TÍTULO}: Detección de células redondas sobre un conjunto de imágenes médicas usando aprendizaje profundo} \\ \hline
	\multicolumn{2}{|p{17cm}|}{\textbf{TITLE}: Round Cell Detection in Medical Images Using Deep Learning} \\ \hline
	\multicolumn{2}{|l|}{\textbf{CONVOCATORIA}: Septiembre, 2025}		\\ \hline
	\multicolumn{2}{|p{17cm}|}{\textbf{RESUMEN}: Con el objetivo de optimizar el análisis seminal, un proceso clave en la medicina reproductiva, 
	se aborda el diseño, desarrollo y validación de un sistema automatizado para la detección de células redondas en imágenes de muestras de semen 
	humano sin procesar, mediante técnicas de aprendizaje profundo y visión por computador.
	
	La metodología se centró en la consolidación de un dataset validado por expertos, el reentrenamiento, la optimización y validación de diversas arquitecturas del modelo YOLO (de la v8 a la v12),
	incluyendo la creación de un modelo personalizado y otro ensamblado. Los resultados obtenidos demuestran que el mejor modelo para abordar esta problemática es el YOLOv12s, 
	superando significativamente el rendimiento de sistemas anteriores, mejorando las métricas clave y reduciendo errores, incluso frente a artefactos visuales.
	
	El proyecto culmina con el desarrollo de una herramienta web interactiva que traduce los diferentes modelos de inteligenia artificial 
	en una solución práctica y funcional, que facilita la validación de resultados por parte del especialitas y sienta las bases para su futura integración en entornos clínicos, potenciando así 
	la transformación digital en el campo de la andrología.} \\ \hline
	\multicolumn{2}{|p{17cm}|}{\textbf{ABSTRACT}: To optimize semen analysis, a key process in reproductive medicine, this work addresses the design, development, and validation of an automated system 
	for detecting round cells in raw human semen sample images, using deep learning and computer vision techniques.
	The methodology focused on the consolidation of an expert-validated dataset, and on the retraining, optimization, and validation of various YOLO model architectures (from v8 to v12), 
	including the creation of custom and ensemble models. The results show that YOLOv12s is the most effective model to address this problem, significantly outperforming previous systems 
	by improving key metrics and reducing errors, even when faced with visual artifacts.
	The project culminates in the development of an interactive web tool that translates the artificial intelligence models into a practical and functional solution. This interface facilitates 
	the validation of results by specialists and lays the groundwork for its future integration into clinical settings, thus promoting the digital transformation in the field of andrology.} \\ 
	\hline
	\multicolumn{2}{|p{17cm}|}{\textbf{Palabras clave}: detección de objetos (\textit{object detection}), células redondas (\textit{round cells}), 
	análisis de semen (\textit{semen analysis}), inteligencia artificial (\textit{artificial intelligence}), aprendizaje profundo (\textit{deep learning}), reproducción asistida (\textit{assisted reproductive technology}), YOLO}	\\ 
	\hline
\end{longtable}