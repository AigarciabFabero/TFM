%%% Documento tipo para trabajos en LaTeX.
%%% Copyleft: Jesús Balsa, Juan F. García.

% Tipo de documento:
\documentclass[12pt,a4paper,onecolumn,oneside]{report}
\newcommand{\mychapter}[2]{
	\setcounter{chapter}{#1}
	\setcounter{section}{0}
	\chapter*{#2}
	\addcontentsline{toc}{chapter}{#2}
}

% Opcional: Tamaño personalizado para los márgenes:
\usepackage[a4paper, top=3cm, bottom=3cm, left=3cm, right=3cm]{geometry}
\usepackage{float}
\usepackage{array}
\usepackage{amsmath}
\usepackage{lscape}
\usepackage{longtable}
\newcommand{\tabtitle}[1]{\cellcolor{CSGreen}\color{white}\textbf{#1}\rule{0pt}{20pt}}
\newcommand{\ltt}[1]{\cellcolor{CSGreen}\color{white}\textbf{#1}\rule{0pt}{8pt}}
\newcommand{\rtabtitle}[1]{\cellcolor{CSRed}\color{white}\textbf{#1}\rule{0pt}{20pt}}
\usepackage{multirow}
\usepackage[utf8]{inputenc} % Codificación UTF-8.
\usepackage[T1]{fontenc}    % Para usar caracteres con tilde.
\usepackage[spanish,es-tabla]{babel} % Escritura en castellano.
\usepackage{eurosym}  % Para el símbolo del EURO (€).
\usepackage{graphicx} % Paquete de imágenes, para introducir figuras.
\DeclareGraphicsExtensions{.pdf,.png,.jpg}
\usepackage[usenames,dvipsnames]{color} % Texto en colores.
\usepackage[table, RGB, usenames,dvipsnames]{xcolor}   % Extra colors.
\definecolor{codegreen}{rgb}{0,0.6,0}
\definecolor{codegray}{rgb}{0.5,0.5,0.5}
\definecolor{codepurple}{rgb}{0.58,0,0.82}
\definecolor{backcolour}{rgb}{0.95,0.95,0.92}
\definecolor{lesslightblue}{rgb}{0.7,0.7,1.0}
\usepackage{listings}
\lstdefinestyle{mystyle}{
    backgroundcolor=\color{backcolour},
    commentstyle=\color{codegreen},
    keywordstyle=\color{magenta},
    stringstyle=\color{codepurple},
    basicstyle=\ttfamily\footnotesize,
    breakatwhitespace=false,
    breaklines=true,
    captionpos=b,
    keepspaces=true,
    showspaces=false,
    showstringspaces=false,
    showtabs=false,
    tabsize=2
}

\lstdefinelanguage{shell}
{
  backgroundcolor=\color{black},
  rulecolor=\color{codegray},
  basicstyle=\footnotesize\ttfamily\color{white},
  morecomment=[l]{\#},
  commentstyle=\footnotesize\ttfamily\color{lesslightblue}
}

\lstset{style=mystyle}

\renewcommand{\lstlistingname}{Script}
\usepackage{url}      % Para escritura de URLs.
\usepackage[breaklinks]{hyperref} % Hiperreferencias.
\usepackage{amsmath,amssymb} % Para los símbolos matemáticos.
\usepackage{cite}     % Para las citas de referencias (crea el superíndice).
\usepackage{listings} % Para coloreado de código fuente.
\usepackage{verbatim} % Para textos tipo consola y otros formatos.
\usepackage{fancyvrb} % Más opciones de verbatim.
\usepackage{parskip}  % OPCIONAL: Separa los párrafos con una línea en blanco.
\setlength{\parindent}{15pt} % Sangría de párrafos estándar (15 puntos). Necesario incluirla si se usa el paquete 'parskip'.
\usepackage[export]{adjustbox}
\usepackage{caption}  % Para personalizar los pies de foto.

% Opciones del paquete caption para los pies de imágenes y tablas:
\captionsetup{figurename=Figura, tablename=Tabla, labelsep=colon, labelfont=bf, font=small, justification=centering}

% Para el control de líneas viudas y huérfanas (líneas sueltas en páginas nuevas):
\usepackage[all]{nowidow}

\usepackage[nottoc]{tocbibind}    % Incluye el apartado "Referencias" en el índice.
%\def\spanishrefname{Bibliografía} % Para que ponga "Bibliografía" en lugar de "Referencias". SÓLO se aplica a formato "article". En "report" ya pone "Bibliografía".

\usepackage{fancyhdr}

\setlength{\unitlength}{1 cm} % Unidad de trabajo de medidas.

\renewcommand*{\baselinestretch}{1.25} % Altura del INTERLINEADO.
\renewcommand{\shorthandsspanish}{}    % Para que corte las palabras según el castellano.
\usepackage{longtable}
\usepackage{dirtree}
\usepackage{multirow }

% Propiedades para el PDF generado (METADATOS):
\newcommand{\authorNames}{Aitor García Blanco}
\newcommand{\pdftitle}{Detección de células defectuosas en imágenes médicas}
\hypersetup{
  pdftitle={\pdftitle},% Título
  pdfauthor={\authorNames},% Autor
  pdfsubject={\pdftitle \ - \authorNames},% Asunto
% pdfkeywords={Opcional: algunas palabras clave}%
}

% Para la representación de código fuente:
% Para BASH:
\lstset{
	language=bash,
	basicstyle=\scriptsize,
	frame=single,
	numbers=left,
	numberstyle=\scriptsize,
	stepnumber=1,
	numbersep=9pt,
	backgroundcolor=\color{White},
	showspaces=false,
	showstringspaces=false,
	showtabs=false,
	tabsize=4,
	captionpos=b,
	breaklines=true,
	keywordstyle=\color{blue}\bfseries,
	%identifierstyle=\color{green}\bfseries,
	stringstyle=\color{orange}\bfseries,
	commentstyle=\color{gray}\bfseries
}
% Para C++:
%\lstset{
%	language=C++,
%	frame=single,
%	keywordstyle=\color{Green}\bfseries,
%	identifierstyle=\color{BlueViolet},
%	stringstyle=\color{Red},
%	commentstyle=\color{MidnightBlue}
%}

\usepackage{fancyhdr} % Para el tamaño y estilo de los encabezados.

\fancypagestyle{headings}{% Redefine el estilo "headings".
	\fancyhf{} % Clear all header and footer fields.
	\lhead{\small \it Máster Universitario en Robótica e Inteligencia Artificial} 
	\rhead{\small \it Página \thepage}      % Nº de página a la derecha. Tamaño "small".
	\renewcommand{\headrulewidth}{1pt}
}

% Ajustes para división manual de palabras:
\hyphenation{Python} % Impide que la palabra Python sea dividida al acabar una línea.

\usepackage{titlesec}
\titleformat{\chapter}[hang]
  {\normalfont\huge\bfseries}
  {\thechapter.}{1em}{}

\titlespacing*{\chapter}{0pt}{0pt}{1em}
%%%%%%%%%%%%%%%%%%%%%%%%%%%%%%%%%%%%%%%%%%%%%%%%%%%%%%%%%%%%%%%%%%%%%%%%%%%%%%%%%%%%%%%%%%%%%%%%%%%%%%%%%%%%%%%%%%%%%
%%%%%%%%%%%%%%%%%%%%%%%%%%%%%%%%%%%%%%%%%% INICIO DEL DOCUMENTO: %%%%%%%%%%%%%%%%%%%%%%%%%%%%%%%%%%%%%%%%%%%%%%%%%%%%
%%%%%%%%%%%%%%%%%%%%%%%%%%%%%%%%%%%%%%%%%%%%%%%%%%%%%%%%%%%%%%%%%%%%%%%%%%%%%%%%%%%%%%%%%%%%%%%%%%%%%%%%%%%%%%%%%%%%%
\renewcommand{\chaptername}{}
\begin{document}

% Página de TÍTULO (portada):
\begin{titlepage}

\begin{picture}(0,0)
\put(-1,-2){\includegraphics[height=3cm]{figuras/logos/logo_master.png}}
\end{picture}

\begin{picture}(0,0)
\put(10,-1.5){\includegraphics[height=3cm]{figuras/logos/logo_ule.png}}
\end{picture}

\begin{center}
\vspace{3cm}
\textbf{{\Large \bf Departamento de Ingenierías}}\\[0.5cm]
\textbf{{\Large \bf Mecánica, Informática y Aeroespacial}}\\[2cm]
{\Large \bf MÁSTER UNIVERSITARIO EN ROBÓTICA E \\ INTELIGENCIA ARTIFICIAL}\\[2.5cm]
{\Large Trabajo de Fin de Máster}\\[2.0cm]
{\Large \textbf{Sistema automatizado para la detección de células redondas en imágenes médicas mediante aprendizaje profundo}\\[0.8cm]} %[0.1cm]} 
%{\Large \textbf{Sistema cognitivo para ROS 2\\[2cm]}}
{\Large \textbf{Automated system for detecting round cells in medical images using deep learning}\\[1.5cm]} %[0.1cm]} 
%{\Large \textbf{Cognitive System for ROS 2\\[2cm]}}
\end{center}

\begin{flushright}
{\bf Autor: Aitor García Blanco}\\[0.3cm]
{\bf Tutor: Laura Fernández Robles}\\[0.3cm]
%{\bf Co-Tutor: }\\[0.3cm]
%{\bf ~}\\[0.5cm]
\end{flushright}


\end{titlepage}

% Página de FIRMAS:
\newpage

\thispagestyle{empty} % para que no se numere esta página.

\begin{center}
%{\huge (Septiembre, 2025)}
\end{center}

\setlength{\LTleft}{-1.25cm}      % Desplaza la tabla 1cm a la izquierda
\setlength{\LTright}{0pt}      % Mantiene el margen derecho normal

\begin{longtable}{|l|l|}
	\cline{1-2}
	\multicolumn{2}{|c|}{}	\\
	\multicolumn{2}{|c|}{\textbf{UNIVERSIDAD DE LEÓN}}	\\
	\multicolumn{2}{|c|}{\textbf{Departamento de Ingenierías Mecánica, Informática y Aeroespacial}}	\\ 
	\multicolumn{2}{|c|}{\textbf{MÁSTER UNIVERSITARIO EN ROBÓTICA E INTELIGENCIA ARTIFICIAL}}	\\
	\multicolumn{2}{|c|}{\textbf{Trabajo de Fin de Máster}}	\\ 
	\multicolumn{2}{|c|}{}	\\ \hline
	\multicolumn{2}{|l|}{\textbf{ALUMNO}: Aitor García Blanco}	\\ \hline
	\multicolumn{2}{|l|}{\textbf{TUTOR}: Laura Fernández Robles}	\\ \hline
	\multicolumn{2}{|p{17cm}|}{\textbf{TÍTULO}: Sistema automatizado para la detección de células redondas en imágenes médicas mediante aprendizaje profundo} \\ \hline
	\multicolumn{2}{|p{17cm}|}{\textbf{TITLE}: Automated system for detecting round cells in medical images using deep learning} \\ \hline
	\multicolumn{2}{|l|}{\textbf{CONVOCATORIA}: Septiembre, 2025}		\\ \hline
	\multicolumn{2}{|p{17cm}|}{\textbf{RESUMEN}: Con el objetivo de optimizar el análisis seminal, un proceso clave en la medicina reproductiva, 
	se aborda el diseño, desarrollo y validación de un sistema automatizado para la detección de células redondas en imágenes de muestras de semen 
	humano sin procesar, mediante técnicas de aprendizaje profundo y visión por computador.
	
	La metodología se centra en la consolidación de un dataset validado por expertos. Posteriormente, se aborda el entrenamiento y la optimización de 
	diversas arquitecturas del modelo YOLO (de la v8 a la v12), incluyendo la creación de un modelo con arquitectura personalizada y otro ensamblado (\textit{ensemble})
	que fusiona las fortalezas de YOLOv10s y YOLOv12s. Para maximizar su rendimiento se aplican técnicas de ajuste fino (\textit{fine-tuning}) y aumento de datos 
	(\textit{data augmentation}), junto con una sistemática optimización de hiperparámetros con Optuna y una evaluación mediante validación cruzada. 
	Los resultados demuestran que el mejor modelo para abordar esta problemática es el YOLOv12s, que supera significativamente el rendimiento de 
	sistemas anteriores.
	
	El proyecto culmina con el desarrollo de una herramienta web interactiva que traduce los diferentes modelos de inteligenia artificial 
	en una solución práctica y funcional, que facilita la validación de resultados por parte del especialitas y sienta las bases para su futura integración en entornos clínicos, potenciando así 
	la transformación digital en el campo de la andrología.} \\ \hline

	\multicolumn{2}{|p{17cm}|}{\textbf{Palabras clave}: detección de objetos, células redondas, 
	análisis de semen, inteligencia artificial, aprendizaje profundo, visión por computador, YOLO}	\\
	\hline

	\multicolumn{2}{|p{17cm}|}{\textbf{ABSTRACT}: To optimize semen analysis, a key process in reproductive medicine, this project addresses the design, 
	development, and validation of an automated system for detecting round cells in images of unprocessed human semen samples using deep learning and 
	computer vision techniques.

	The methodology focused on the consolidation of an expert-validated dataset. Subsequently, various YOLO model architectures (from v8 to v12) were trained 
	and optimized, including the creation of a custom architecture model and an ensemble model that combines the strengths of YOLOv10s and YOLOv12s. 
	To maximize performance, techniques such as fine-tuning and data augmentation were applied, along with systematic hyperparameter optimization using Optuna 
	and evaluation through cross-validation. The results demonstrate that the best model to address this problem is YOLOv12s, which significantly outperforms 
	previous systems.

	The project culminates in the development of an interactive web tool that translates the artificial intelligence models into a practical and 
	functional solution. This tool facilitates the validation of results by specialists and lays the groundwork for its future integration into 
	clinical environments, thereby advancing the digital transformation in the field of andrology.} \\ 
	\hline
	\multicolumn{2}{|p{17cm}|}{\textbf{Keywords}: object detection, round cells, 
	semen analysis, artificial intelligence, deep learning, computer vision, YOLO}	\\ 
	\hline
\end{longtable}

\newpage
\pagestyle{plain}

\renewcommand{\thepage}{\roman{page}}
\setcounter{page}{1} % Esta página es la 1.

% Página con el ÍNDICE GENERAL
\renewcommand{\contentsname}{Índice}
\tableofcontents

% Página con el ÍNDICE DE FIGURAS
\listoffigures

% Página con el ÍNDICE DE TABLAS
\listoftables

% Página con el GLOSARIO:
\mychapter{0}{Glosario de términos}
\label{chap:glosario}
% A partir de aquí ya se incluye el encabezado en las páginas:
\pagestyle{headings}

% Al ser un capítulo sin número, hay que indicarle qué título añadir al encabezado de la página:
\markboth{GLOSARIO}{} 

\begin{description}
	
	\item[ciberseguridad]: Protección de los sistemas informáticos y de sus redes de comunicaciones, con el objetivo de mantener segura la información que procesan.

\end{description} 

%%%%%%%%%%%%%%%%%%%%%%%%%%%%%%%%%%%%%%%%%%%%%%%%%%%%%%%%%%%%%%%%%%%%%%%%%%%%%%%%%%%%%%%%%%%%%%%%%%%%%%%%%%%%%%%%%%%%%
%%%%                                           Inicio de los CAPÍTULOS                                           %%%%
%%%%%%%%%%%%%%%%%%%%%%%%%%%%%%%%%%%%%%%%%%%%%%%%%%%%%%%%%%%%%%%%%%%%%%%%%%%%%%%%%%%%%%%%%%%%%%%%%%%%%%%%%%%%%%%%%%%%%
\newpage
\renewcommand{\thepage}{\arabic{page}}
\setcounter{page}{1} % Esta página es la 1.

\chapter{Introducción}
\section{Contexto y motivación}
\section{Definición del problema}
\section{Estado del arte e hipótesis}
\section{Objetivos}

En este Trabajo de Fin de Máster, se plantean una serie de objetivos interconectados que tienen como finalidad principal:

\fbox{\begin{minipage}{14cm}
Objetivo principal
\end{minipage}}

De este modo, se pretenden abordar los siguientes objetivos: 

\begin{itemize}
 

    \item[$O_1$]

    \item[$O_2$] 
    
    \item[$O_3$] 
\end{itemize}



\section*{Metodología}


\section*{Tecnologías utilizadas}



\section*{Estructura del trabajo}



\chapter{Antecedentes}
\label{Antecedentes}

En este capítulo, se establece el contexto del estudio explorando las tecnologías, plataformas, herramientas y trabajos previos relacionados que sirven como base para la investigación.

\section{Contexto médico}

\section{Visión por computador}
\section{Arquitecturas de \textit{Deep Learning}}
\subsection{Modelos de una etapa: YOLO}
\subsection{Modelos de dos estapas: Fast RCNN}

\section{Técnicas avanzadas para la mejora de modelos}
\subsection{Aumento de datos (\textit{Data Augmentation})}
\subsection{Optimización de hiperparámetros}
\subsection{Validación cruzada}
\subsection{Ensablado de modelos}


\chapter{Gestión de proyecto software}
\label{Gestión de proyecto software}



\section{Alcance del proyecto}
\label{Alcance del proyecto}

\subsection{Definición del proyecto}



\subsection{Estimación de tareas y recursos}



\subsection{Presupuesto}


\subsubsection{Coste de personal}


\subsubsection{Coste del hardware}



\subsubsection{Coste total}

\section{Plan de trabajo}
\label{Plan de trabajo}


\subsection{Identificación de tareas}
\label{Identificación de tareas}



\subsection{Estimación de tareas}
\label{Estimación de tareas}

\subsection{Planificación de tareas}
\label{Planificación de tareas}



\section{Gestión de recursos}
\label{Gestión de recursos}

\subsection{Especificación de recursos}
\label{Especificación de recursos}

\subsection{Asignación de recursos}
\label{asignacion de recursos}



\section{Gestión de riesgos}
\subsection{Identificación de riesgos}
\subsection{Análisis de riesgos}

\section{Legislación y normativa}

En el marco de ejecución de este proyecto, se ha llevado a cabo un riguroso cumplimiento de la legislación y normativa vigente. A continuación, se detalla cómo el proyecto se ajusta y adhiere a las leyes pertinentes:

\begin{itemize}
    \item \textbf{Ley Orgánica 3/2018, de 5 de diciembre, de Protección de Datos Personales y garantía de los derechos digitales}\cite{LOPD2018}
    
    Este proyecto respeta plenamente la Ley Orgánica 3/2018, la cual reconoce el derecho fundamental a la protección de datos personales. La creación del dataset se ha desarrollado de acuerdo con las disposiciones de la ley, asegurando la legalidad en el tratamiento de datos. Los permisos explícitos obtenidos para el uso de imágenes garantizan la privacidad de los individuos. Además, las pruebas y el desarrollo se realizaron en un entorno controlado para evitar perjuicios en materia de datos a terceros. Aunque las imágenes capturadas por la cámara del robot no se almacenan, se garantiza que el procesamiento cumple con los principios de protección de datos establecidos en la ley.
    \item \textbf{Reglamento (UE) 2016/679 del Parlamento Europeo y del Consejo relativo a la protección de las personas físicas en lo que respecta al tratamiento de datos personales y a la libre circulación de estos datos (RGPD)}\cite{RGPD2016}

    La creación del dataset en este proyecto se ajusta a los principios del RGPD, asegurando la legalidad y transparencia en el tratamiento de datos personales. Se han implementado medidas técnicas y organizativas para garantizar la seguridad y privacidad de los datos presentes en el dataset, cumpliendo así con las normativas del RGPD.
    \item \textbf{Propuesta de Reglamento del Parlamento Europeo y del Consejo sobre Inteligencia Artificial (PRLIA)}\cite{PRIA2021}

    El proyecto ha sido diseñado y ejecutado considerando de manera integral las disposiciones y requisitos establecidos por la PRLIA. Un documento legislativo propuesto por la Unión Europea para regular el uso de la inteligencia artificial en diversos ámbitos. Esta propuesta tiene como objetivo establecer normas armonizadas en materia de inteligencia artificial para garantizar la seguridad, ética y transparencia en su desarrollo y aplicación. Aunque el proyecto no involucra una IA de alto riesgo, se han tenido en cuenta las recomendaciones éticas reflejadas en el marco ético del proyecto. No se ha incumplido ninguna disposición de la PRLIA, ya que no se han violado derechos fundamentales ni se han utilizado técnicas subliminales o vulnerabilidades de grupos específicos.
    \item \textbf{Real Decreto Legislativo 1/1996 sobre Propiedad Intelectual}\cite{RDL1996}
    
    En conformidad con el Real Decreto Legislativo 1/1996, el proyecto respeta la normativa sobre propiedad intelectual. Se ha optado por utilizar únicamente código y herramientas de software libre y de código abierto para garantizar el cumplimiento de la normativa en materia de propiedad intelectual.
\end{itemize}

\chapter{Experimentación}
\label{Experimentación}

\section{Dataset}


\section{Entorno de desarrollo}
\subsection{\textit{Hardware}}
\subsection{\textit{Software}}

\section{Configuraciones}
\subsection{Entrenamiento}
\subsection{Validación cruzada}
\subsection{Ensamblado de modelos}

\section{Métricas}
\subsection{\textit{Precision}}
\subsection{\textit{Recall}}
\subsection{\textit{mean Average Precision}}

\arrayrulecolor[HTML]{B9DAE1}
% \begin{longtable} {|p{3cm}|p{11cm}|} \hline
% \rowcolor[HTML]{67D1E6} CU-01 & Capturar imágenes de la cámara  \\ \hline
% Versión & 1.0 \\ \hline
% Descripción & El sistema se subscribe al topic de ROS 2 encargado de publicar
% las imágenes en tiempo real de la cámara del robot o cualquier
% cámara externa conectada al ordenador.\\ \hline
% Actores &  Nodo Detector\\ \hline
% Precondición &  La cámara debe estar publicando en el topic de ROS 2.\\ \hline
% Flujo básico & 
% \begin{tabular}{p{2cm}|p{8cm}}
%  Paso & Acción \\ \hline
%  1 & Se inicializa el nodo Detector.\\ \hline
%  2 &  Se cargan los parámetros indicados por el usuario.\\ \hline
%  3 &  El nodo se subscribe al topic de ROS en el que está publicando la cámara.\\ \hline
% \end{tabular}\\ \hline
% Flujo alternativo &  \\ \hline
% Postcondición &  El nodo Detector recibe las imágenes de la cámara en tiempo real.\\ \hline
% \caption{Caso de uso 1}
% \end{longtable} \vspace{0.5cm}

\chapter{Resultados}
\label{Resultados}

\section{Resultados de los modelos}
\subsection{YOLO}
\subsection{Fast-RCNN}
\subsection{Ensemble}
\subsection{Modelo personalizado}

\section{Discursión de resultados}

\chapter{Herramienta Web}
\section{Objetivos de la herramienta}
\section{Arquitectura (Frontend, Backend)}

\chapter{Conclusión}
\section{Conclusión general}
\section{Limitaciones del estudio}
\section{Lineas de trabajo futuro}

\subsection*{Aportaciones realizadas}



\subsection*{Problemas encontrados}


\section*{Opiniones personales}


\subsection*{Trabajos futuros}


\subsection*{Agradecimientos}



%%%% BIBLIOGRAFÍA %%%%
\renewcommand\bibname{Bibliografía}

%\bibliography{Refs-proyecto}    % Fichero con las referencias.
%Información bibliográfica citada en el texto del trabajo. Otras lecturas recomendadas o consultadas, de figurar, aparecerán en anexos.
%Se debe seguir la norma ISO 690 (buscar en google ISO 690 ugr)

% % También se puede optar por incluir las referencias desde aquí mismo:
% \begin{thebibliography}{1}
	

	

% \bibitem{CSSP}
% {CSSP, D.}:
% \newblock \textit{Recommended Practice: Improving Industrial Control Systems
%   Cybersecurity with Defense-In-Depth Strategies}.
% \newblock US-CERT Defense In Depth. (Octubre 2009)

% \end{thebibliography}

\bibliographystyle{plain} % Fichero con el formato de la bibliografía.
\nocite{*}
\bibliography{referencias}
%\bibliographystyle{unsrt} %plain %apalike


%%%% ANEXOS %%%%
\renewcommand{\appendixname}{Anexo}
\titleformat{\chapter}[display]
  {\normalfont\huge\bfseries}
  {Anexo \thechapter}{0pt}{\Huge}
\appendix

\chapter{Control de versiones}
\label{Control de versiones}

En el marco del desarrollo de este Trabajo de Fin de Máster (TFM), se ha empleado GitHub como servicio de control de versiones para gestionar eficientemente el código generado. GitHub ha proporcionado una plataforma robusta que facilita la colaboración, el seguimiento de cambios y la gestión de versiones de manera integral.


\chapter{Seguimiento de proyecto fin de máster}
\label{Seguimiento de proyecto fin de máster}

%Obligatorio. Seguimiento del trabajo real.

\section{Forma de seguimiento}

\section{Planificación inicial}

\section{Planificación final}
%Si el trabajo ha consistido en la elaboración de una aplicación se incluirá el manual de usuario de la misma.

\chapter{Herramienta Web}
\section{Manual de usuario}


\end{document}



% Antes de experimentación, añadir un section{metodología} con los 4 pipelines para cada metodología.
% el punto 4.3.3 si hacemos lo anterior lo quitamos.
% En experimentación hablar de las métricas
% sección nueva de la herramienta web después de resultados. El punto C 