\begin{center}
%{\huge (Septiembre, 2025)}
\end{center}

\setlength{\LTleft}{-1.25cm}      % Desplaza la tabla 1cm a la izquierda
\setlength{\LTright}{0pt}      % Mantiene el margen derecho normal

\begin{longtable}{|l|l|}
	\cline{1-2}
	\multicolumn{2}{|c|}{}	\\
	\multicolumn{2}{|c|}{\textbf{UNIVERSIDAD DE LEÓN}}	\\
	\multicolumn{2}{|c|}{\textbf{Departamento de Ingenierías Mecánica, Informática y Aeroespacial}}	\\ 
	\multicolumn{2}{|c|}{\textbf{MÁSTER UNIVERSITARIO EN ROBÓTICA E INTELIGENCIA ARTIFICIAL}}	\\
	\multicolumn{2}{|c|}{\textbf{Trabajo de Fin de Máster}}	\\ 
	\multicolumn{2}{|c|}{}	\\ \hline
	\multicolumn{2}{|l|}{\textbf{ALUMNO}: Aitor García Blanco}	\\ \hline
	\multicolumn{2}{|l|}{\textbf{TUTOR}: Laura Fernández Robles}	\\ \hline
	\multicolumn{2}{|p{17cm}|}{\textbf{TÍTULO}: Sistema automatizado para la detección de células redondas en imágenes médicas mediante aprendizaje profundo} \\ \hline
	\multicolumn{2}{|p{17cm}|}{\textbf{TITLE}: Automated system for detecting round cells in medical images using deep learning} \\ \hline
	\multicolumn{2}{|l|}{\textbf{CONVOCATORIA}: Septiembre, 2025}		\\ \hline
	\multicolumn{2}{|p{17cm}|}{\textbf{RESUMEN}: Con el objetivo de optimizar el análisis seminal, un proceso clave en la medicina reproductiva, 
	se aborda el diseño, desarrollo y validación de un sistema automatizado para la detección de células redondas en imágenes de muestras de semen 
	humano sin procesar, mediante técnicas de aprendizaje profundo y visión por computador.
	
	La metodología se centra en la consolidación de un dataset validado por expertos. Posteriormente, se aborda el entrenamiento y la optimización de 
	diversas arquitecturas del modelo YOLO (de la v8 a la v12), incluyendo la creación de un modelo con arquitectura personalizada y otro ensamblado (\textit{ensemble})
	que fusiona las fortalezas de YOLOv10s y YOLOv12s. Para maximizar su rendimiento se aplican técnicas de ajuste fino (\textit{fine-tuning}) y aumento de datos 
	(\textit{data augmentation}), junto con una sistemática optimización de hiperparámetros con Optuna y una evaluación mediante validación cruzada. 
	Los resultados demuestran que el mejor modelo para abordar esta problemática es el YOLOv12s, que supera significativamente el rendimiento de 
	sistemas anteriores.
	
	El proyecto culmina con el desarrollo de una herramienta web interactiva que traduce los diferentes modelos de inteligenia artificial 
	en una solución práctica y funcional, que facilita la validación de resultados por parte del especialitas y sienta las bases para su futura integración en entornos clínicos, potenciando así 
	la transformación digital en el campo de la andrología.} \\ \hline

	\multicolumn{2}{|p{17cm}|}{\textbf{Palabras clave}: detección de objetos, células redondas, 
	análisis de semen, inteligencia artificial, aprendizaje profundo, visión por computador, YOLO}	\\
	\hline

	\multicolumn{2}{|p{17cm}|}{\textbf{ABSTRACT}: To optimize semen analysis, a key process in reproductive medicine, this project addresses the design, 
	development, and validation of an automated system for detecting round cells in images of unprocessed human semen samples using deep learning and 
	computer vision techniques.

	The methodology focused on the consolidation of an expert-validated dataset. Subsequently, various YOLO model architectures (from v8 to v12) were trained 
	and optimized, including the creation of a custom architecture model and an ensemble model that combines the strengths of YOLOv10s and YOLOv12s. 
	To maximize performance, techniques such as fine-tuning and data augmentation were applied, along with systematic hyperparameter optimization using Optuna 
	and evaluation through cross-validation. The results demonstrate that the best model to address this problem is YOLOv12s, which significantly outperforms 
	previous systems.

	The project culminates in the development of an interactive web tool that translates the artificial intelligence models into a practical and 
	functional solution. This tool facilitates the validation of results by specialists and lays the groundwork for its future integration into 
	clinical environments, thereby advancing the digital transformation in the field of andrology.} \\ 
	\hline
	\multicolumn{2}{|p{17cm}|}{\textbf{Keywords}: object detection, round cells, 
	semen analysis, artificial intelligence, deep learning, computer vision, YOLO}	\\ 
	\hline
\end{longtable}